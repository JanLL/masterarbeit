\documentclass{beamer}

\usepackage[english]{babel}
\usepackage[utf8x]{inputenc}
\usepackage{amsmath,amsfonts,amssymb}


\usepackage{subcaption}

\usepackage[absolute,overlay]{textpos}





\newcommand{\todo}[1]{\textcolor{red}{TODO: #1}}

\addtobeamertemplate{navigation symbols}{}{%
    \usebeamerfont{footline}%
    \usebeamercolor[fg]{footline}%
    \hspace{1em}%
    \insertframenumber/\inserttotalframenumber
}

\setbeamertemplate{bibliography item}[text]
\bibliographystyle{unsrt}


\usetheme{Luebeck}
\usecolortheme{orchid}

\title{Current state of master thesis}
\subtitle{Simulation of DSC measuring process and \\ parameter estimation of specific heat capacity $c_p$}
\author{Jan Lammel}

\begin{document}
	
\frame{\titlepage}

\frame{
	\frametitle{Table of contents}
	\tableofcontents[hideallsubsections]
	
}

\section{Introduction}
\frame{
\frametitle{Introduction}

}

\section{Physics}
\subsection{Physical quantities}
\subsection{Differential Scanning Calorimetry (DSC)}

\section{Parameter Estimation}
\subsection{Mathematical model}
\subsection{Parametrization}
\subsection{Results}
\frame{
	\frametitle{Results}

	\begin{itemize}
		\item Waermestrom-Fit Residuum + zugehoeriges $c_p$ fuer vllt 2, 3 versch. Heizraten
		\item In einem plot alle $c_p$ Kurven fuer alle Heizraten um Shift zu zeigen
	\end{itemize}
	
}


\section{Next potential steps}
\frame{
	\frametitle{Next potential steps}
	
	\begin{itemize}
		\item Heat equation with $\rho=\rho(T)$ \\
		$\rightarrow$ problem with value of absolute enthalpy term
		\item $L_1$ and $L_3$ as optimization variable. \\
		$\rightarrow$ huge programming effort
	\end{itemize}
	
}



\section{Open Questions}
\frame{
	\frametitle{Open Questions}
	
	\begin{itemize}
		\item Heat equation derivation with $\rho=\rho(T)$ correct?
	\end{itemize}
	
}





	
	
\end{document}