\documentclass{scrartcl}[12pt, halfparskip]
\usepackage[utf8x]{inputenc}
\usepackage[T1]{fontenc}
\usepackage[english]{babel}
\usepackage{amsmath, amssymb, amstext}
\usepackage{wrapfig}
\usepackage{float}
\usepackage{graphicx}
\usepackage{color}


\usepackage{caption}
\usepackage{subcaption}

\bibliographystyle{unsrt}

\setlength{\parindent}{0pt}

\newcommand{\todo}[1]{\textcolor{red}{TODO: #1}}
\newcommand{\bem}[1]{\textcolor{blue}{Bem.: #1}}


\title{Master Thesis}


\author{Jan Lammel}
\date{\today{}, Heidelberg}



\begin{document}

\begin{titlepage}
	\begin{center}
	
	\textsc{\large Ruprecht-Karls-Universit\"{a}t Heidelberg} \\[0.5cm]
	\textsc{\large Master Thesis}\\[1cm]
	
	\newcommand{\HRule}{\rule{\linewidth}{0.5mm}}
	\HRule \\[0.4cm]
	\huge \bfseries Ueberschrift
	\HRule 
	
	\vspace{11cm}
	
	\Large \textit{Jan Lammel }\\
	\Large \textit{Heidelberg, \today }\\ \vspace{0.5cm}
	\Large \textit{Supervisor:}
	
	\Large \textit{Prof. Dr. Dr. h. c. mult. Hans Georg Bock}\\
	\Large \textit{Dr. Andreas Sommer}
	
	\end{center}
\end{titlepage}

\newpage

\tableofcontents 
\newpage

 \pagenumbering{Roman} 
 

 \addsec{Abstract}
 bla bla bla
 
 
 \newpage

\addsec{Zusammenfassung}
blubb !

\newpage
\pagenumbering{arabic}
\section{Introduction}

\section{Physical Background}
\subsection{Heat equation}
As we are dealing with heat transport the central equation in this thesis is the



\subsection{Differential Scanning Calorimetry}




\newpage
\section{Mathematical Background}
\subsection{Numerical integration of ODE systems}


\subsection{Derivative generation}
In the context of continuous optimization we are dependent on derivative information in order to minimize some function. There are several possibilities to gain these derivatives. We will introduce here the concept of finite differences and automatic differentiation which were used in this thesis. Beside them the methods of symbolic- and complex step-differentiation exist and are explained in detail in \cite{diss_jan}.
\subsubsection{Finite Differences}
\bem{Wichtig weil wir am Anfang mit den NURBS noch finite Differenzen benutzt haben... ausserdem sind finite Differenzen auch fuer Linienmethode bei der PDE Diskretisierung notwendig.}\\

Derivatives generated by finite differences can be derived easily from Taylor-series expansion. Considering the function $f: \mathcal{R} \rightarrow \mathcal{R}^m$ the Taylor series looks as follows:
\begin{equation}
	f(x+h) = f(x) + h \cdot \frac{\partial f}{\partial x}(x) + \mathcal{O}(h^2)
\end{equation}

Reordering gives the one-sided derivative approximation

\begin{equation}
	\frac{\partial f}{\partial x}(x) = \frac{f(x+h) - f(x)}{h} + \mathcal{O}(h^2)
\end{equation}

Regarding the Taylor series up to order two in both directions of the domain of definition,

\begin{align}
	f(x+h) = f(x) + h \cdot \frac{\partial f}{\partial x}(x) + \frac{h^2}{2} \cdot \frac{\partial^2 f}{\partial^2 x}(x) + \mathcal{O}(h^3) \\
	f(x-h) = f(x) - h \cdot \frac{\partial f}{\partial x}(x) + \frac{h^2}{2} \cdot \frac{\partial^2 f}{\partial^2 x}(x) + \mathcal{O}(h^3)	
\end{align}

subtract both equations and reorder again we get

\begin{equation}
	\frac{\partial f}{\partial x}(x) = \frac{f(x+h) - f(x-h)}{2 h^2} + \mathcal{O}(h^3)
\end{equation}

which has a higher error order but at the expense of two function evaluations.


\subsubsection{Automatic Differentiation and IND}



\subsection{Optimization Task: Parameter Estimation}


\newpage
\section{Simulation of DSC measuring process and parameter estimation of specific heat capacity $c_p$}
\subsection{Analytical solution of reference side}
\subsection{Mathematical model}
\subsection{Parametrizations of $c_p$}
\subsection{Influence of spatial discretization grid}




\begin{thebibliography}{9}

\bibitem{diss_jan}
	 Adjoint-based algorithms and numerical methods for sensitivity generation and optimization of large scale dynamic systems, 
	 Jan Albersmeyer, 2010,
	 http://www.ub.uni-heidelberg.de/archiv/11651

  
\end{thebibliography}

\end{document}