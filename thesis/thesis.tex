\documentclass{scrartcl}[12pt, halfparskip]
\usepackage[utf8x]{inputenc}
\usepackage[T1]{fontenc}
\usepackage[english]{babel}
\usepackage{amsmath, amssymb, amstext}
\usepackage{wrapfig}
\usepackage{float}
\usepackage{graphicx}
\usepackage{color}


\usepackage{caption}
\usepackage{subcaption}

\bibliographystyle{unsrt}

\setlength{\parindent}{0pt}

\newcommand{\todo}[1]{\textcolor{red}{TODO: #1}}
\newcommand{\bem}[1]{\textcolor{blue}{Bem.: #1}}


\title{Master Thesis}


\author{Jan Lammel}
\date{\today{}, Heidelberg}



\begin{document}

\begin{titlepage}
	\begin{center}
	
	\textsc{\large Ruprecht-Karls-Universit\"{a}t Heidelberg} \\[0.5cm]
	\textsc{\large Master Thesis}\\[1cm]
	
	\newcommand{\HRule}{\rule{\linewidth}{0.5mm}}
	\HRule \\[0.4cm]
	\huge \bfseries Ueberschrift
	\HRule 
	
	\vspace{11cm}
	
	\Large \textit{Jan Lammel }\\
	\Large \textit{Heidelberg, \today }\\ \vspace{0.5cm}
	\Large \textit{Supervisor:}
	
	\Large \textit{Prof. Dr. Dr. h. c. mult. Hans Georg Bock}\\
	\Large \textit{Dr. Andreas Sommer}
	
	\end{center}
\end{titlepage}

\newpage

\tableofcontents 
\newpage

 \pagenumbering{Roman} 
 

 \addsec{Abstract}
 bla bla bla
 
 
 \newpage

\addsec{Zusammenfassung}
blubb !

\newpage
\pagenumbering{arabic}
\section{Introduction}

\section{Physical Background}
\subsection{Heat equation}
As we are dealing with heat transport the central equation in this thesis is the



\subsection{Differential Scanning Calorimetry}




\newpage
\section{Mathematical Background}
\subsection{Numerical integration of ODE systems}
\subsection{PDE discretization: Method of lines}

\subsection{Derivative generation}
In the context of continuous optimization we are dependent on derivative information in order to minimize some function. There are several possibilities to gain these derivatives. We will introduce here the concept of finite differences and automatic differentiation which were used in this thesis. Beside them the methods of symbolic- and complex step-differentiation exist and are explained in detail in \cite{diss_jan}.
\subsubsection{Finite Differences}
\bem{Wichtig weil wir am Anfang mit den NURBS noch finite Differenzen benutzt haben... ausserdem sind finite Differenzen auch fuer Linienmethode bei der PDE Diskretisierung notwendig.}\\

Derivatives generated by finite differences can be derived easily from Taylor-series expansion. Considering the function $f: \mathcal{R} \rightarrow \mathcal{R}^m$ the Taylor series looks as follows:
\begin{equation}
	f(x+h) = f(x) + h \cdot \frac{\partial f}{\partial x}(x) + \mathcal{O}(h^2)
\end{equation}

Reordering gives the one-sided derivative approximation

\begin{equation}
	\frac{\partial f}{\partial x}(x) = \frac{f(x+h) - f(x)}{h} + \mathcal{O}(h^2)
\end{equation}

Regarding the Taylor series up to order two in both directions of the domain of definition,

\begin{subequations}
\label{eq:finite_differences_taylor_exp}
\begin{align}
	f(x+h) = f(x) + h \cdot \frac{\partial f}{\partial x}(x) + \frac{h^2}{2} \cdot \frac{\partial^2 f}{\partial^2 x}(x) + \mathcal{O}(h^3) \label{eq:finite_differences_taylor_exp_+} \\
	f(x-h) = f(x) - h \cdot \frac{\partial f}{\partial x}(x) + \frac{h^2}{2} \cdot \frac{\partial^2 f}{\partial^2 x}(x) + \mathcal{O}(h^3)  \label{eq:finite_differences_taylor_exp_-}	
\end{align}
\end{subequations}


subtract both equations and reorder again we get

\begin{equation}
	\frac{\partial f}{\partial x}(x) = \frac{f(x+h) - f(x-h)}{2 h^2} + \mathcal{O}(h^3)
\end{equation}

which has a higher error order but at the expense of two function evaluations. \\

Second order derivatives can be gained analogously by adding \eqref{eq:finite_differences_taylor_exp_+} and \eqref{eq:finite_differences_taylor_exp_-}:

\begin{equation}
	\frac{\partial^2 f}{\partial^2 x}(x) = \frac{f(x-h) - 2 \cdot f(x) + f(x+h)}{h^2} + \mathcal{O}(h^3)
	\label{eq:finite_difference_2nd_der}
\end{equation}


\todo{Hier noch ne kleine Skizze einfuegen mit dem Diskretisierungsschema mit $\alpha$, und text vervollstaendigen} \\
In the spatial discretization of partial differential equations one often choose a finer grid for areas ...

\begin{subequations}
	\label{eq:finite_differences_taylor_exp_non-homogenous}
	\begin{align}
	f(x-h) = & f(x) - h \cdot \frac{\partial f}{\partial x}(x) + \frac{h^2}{2} \cdot \frac{\partial^2 f}{\partial^2 x}(x) + \mathcal{O}(h^3) \label{eq:finite_differences_taylor_exp_non-homogenous_1} \\
	f(x+\alpha h) = & f(x) - \alpha h \cdot \frac{\partial f}{\partial x}(x) + \frac{\alpha^2 h^2}{2} \cdot \frac{\partial^2 f}{\partial^2 x}(x) + \mathcal{O}(h^3)  \label{eq:finite_differences_taylor_exp_non-homogenous_2}
	\end{align}
\end{subequations}

Multiplying \eqref{eq:finite_differences_taylor_exp_non-homogenous_1} with $\alpha$ and adding \eqref{eq:finite_differences_taylor_exp_non-homogenous_2} gives

\begin{align}
	\alpha f(x-h) + f(x+\alpha h) = \alpha f(x) + \alpha \frac{h^2}{2} \frac{\partial^2 f}{\partial^2 x}(x) + f(x) + \frac{\alpha^2 h^2}{2} \frac{\partial^2 f}{\partial^2 x}(x) \\
	\Leftrightarrow (\alpha+1) \frac{\alpha h^2}{2} \frac{\partial^2 f}{\partial^2 x}(x) = \alpha f(x-h) - (\alpha+1) f(x) + f(x+\alpha h)
\end{align}

\begin{equation}
	\Leftrightarrow \frac{\partial^2 f}{\partial^2 x}(x) = \frac{1}{h^2} \left[ \frac{2}{1+\alpha} f(x-h) - \frac{2}{\alpha} f(x) + \frac{2}{\alpha (\alpha+1)} f(x+\alpha h) \right]
\end{equation}

Note that for a homogeneous grid ($\alpha=1$) this is obviously equal to \eqref{eq:finite_difference_2nd_der}. \\

For simplicity we restricted the function argument $x \in \mathcal{R}$. In the case $x \in \mathcal{R}^n$ the Jacobian is gained by performing directional derivatives. So e.g. for the one-sided first derivative $n+1$ function evaluations are necessary.

\todo{Vor-/Nachteile...} 



\subsubsection{Automatic Differentiation and Internal Numerical Differentiation}
The basic idea of Automatic Differentiation (AD) is to subdivide a function $f: \mathcal{R}^n \rightarrow \mathcal{R}^m$ into so called elementary functions $\varphi_i$ from which the derivative is known. 

The evaluation of these elementary functions give intermediate values $v_i = \varphi(v_j)_{j \prec i}$ where the dependency relation $\prec$ is defined as

\begin{equation}
	j \prec i \Leftrightarrow v_j \text{\textit{ is an argument of }} \varphi_i.
\end{equation}

By successively applying the chain rule one obtains the derivative of $f$.
One can distinguish the forward and reverse mode which will be explained in more detail now and exemplified by means of the example function

\begin{equation}
	F(x) = 
	\begin{bmatrix}
	\exp((1+x_1)^2) + x_3 \\
	x_2 \cdot \sin(1+x_1)
	\end{bmatrix}
\end{equation}

\paragraph{Forward Mode}\mbox{}\\
In the forward mode the directional derivative $\dot{y}$ is computed for a given direction in the function arguments $\dot{x}$ at a evaluation point $x$:

\begin{equation}
	\dot{y} = \frac{\partial f}{\partial x}(x) \cdot \dot{x}
	\label{eq:AD_example}
\end{equation}

The algorithm called first order forward sweep is structured into three parts. First the auxiliary variables $v_{1-n},...,v_0$ and $\dot{v}_{1-n},...,\dot{v}_0$ are initialized with the evaluation point $x$ and the direction of the directional derivative $\dot{x}$. So if we want to get $\frac{\partial f}{\partial x_2}$ we would need to set $\dot{x} = \begin{bmatrix}
0 & 1 & 0 & \dots & 0
\end{bmatrix}^T$. One can see here that with one forward sweep we get one column of the Jacobian $ \frac{\partial f}{\partial x}$. 
After the initialization the actual forward sweep begins where the $k$ elemental functions of $f$ are evaluated and saved in the intermediate values $v_i$. Simultaneously their derivatives  $\dot{v}_i$ are computed using previously calculated intermediate values. E.g. for $v_2 = \varphi_2(v_0, v_1)$ it would hold $\dot{v}_2 = \frac{\partial \varphi_2}{\partial v_0} \dot{v_0} + \frac{\partial \varphi_2}{\partial v_1} \dot{v_1}$. \\
The last $m$ intermediate variables represent the solution vector. So finally this values and derivatives are extracted.


\todo{Q: Ist das hier eine totale Ableitung?}


\begin{table}[H]
\begin{tabular}{|c | l c l | l |} \hline
	 Initialization & $[v_{i-n}, \dot{v}_{i-n}]$ & $=$ & $[x_i, \dot{x}_i]$ & $i=1,...,n$ \\ \hline
	Intermediate steps & $[v_{i}, \dot{v}_{i}]$ & $=$ & $[\varphi_i(v_j)_{j \prec i}, \sum_{j \prec i} \frac{\partial \varphi_i}{\partial v_j}(v_j) \cdot \dot{v}_j]$ & $i=1,...,k$ \\ \hline
	Extract solution & $[y_{m-i}, \dot{y}_{m-i}]$ & $=$ & $[v_{k-i}, \dot{v}_{k-i}]$ & $i=m-1,...,0$ \\ \hline
\end{tabular}
\caption{First order forward sweep}
\label{tab:first_order_forward_sweep}
\end{table}

In order to illustrate this algorithm we will apply it onto the example function \eqref{eq:AD_example}:

\begin{table}[H]
\centering
\begin{tabular}{| c | l | l |} \hline
	Initialization & $v_{-2} = x_1$ & $\dot{v}_{-2} = \dot{x_1}$ \\
	& $v_{-1} = x_2$ & $\dot{v}_{-1} = \dot{x}_2$ \\
	& $v_{0} = x_3$ & $\dot{v_{0}} = \dot{x}_3$ \\ \hline
	Intermediate Steps & $v_1 = 1+v_{-2}$ & $\dot{v_1} = 1 \cdot \dot{v}_{-2}$ \\
	& $v_2 = v_{1}^2$ & $\dot{v_2} = 2 v_1 \cdot \dot{v}_{1}$ \\
    & $v_3 = \exp(v_{2})$ & $\dot{v_3} = \exp(v_2) \cdot \dot{v}_{2}$ \\
    & $v_4 = \sin(v_{1})$ & $\dot{v_4} = \cos(v_1) \cdot \dot{v}_{1}$ \\
	& $v_{5} = v_3 + v_0$ & $\dot{v_{5}} = 1 \cdot \dot{v}_3 + 1 \cdot \dot{v}_0$ \\
	& $v_{6} = v_{-1} + v_4$ & $\dot{v_{6}} = v_4 \cdot \dot{v}_{-1} + v_4 \cdot \dot{v}_{-1}$ \\ \hline
	Extract  solution & $y_1 = v_5$ & $\dot{y}_1 = \dot{v}_5$ \\
	& $y_2 = v_6$ & $\dot{y}_2 = \dot{v}_6$ \\ \hline
\end{tabular}
\caption{First order forward sweep applied on \eqref{eq:AD_example}.}
\end{table}




\paragraph{Adjoint Mode}\mbox{}\\



\subsection{Optimization Task: Parameter Estimation}


\newpage
\section{Simulation of DSC measuring process and parameter estimation of specific heat capacity $c_p$}
\subsection{Analytical solution of reference side}
\subsection{Mathematical model}
\subsection{Parametrizations of $c_p$}
\subsection{Influence of spatial discretization grid}




\begin{thebibliography}{9}

\bibitem{diss_jan}
	 Adjoint-based algorithms and numerical methods for sensitivity generation and optimization of large scale dynamic systems, 
	 Jan Albersmeyer, 2010,
	 http://www.ub.uni-heidelberg.de/archiv/11651

  
\end{thebibliography}

\end{document}