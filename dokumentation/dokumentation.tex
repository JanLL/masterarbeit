\documentclass{scrartcl}[12pt, halfparskip]
\usepackage[utf8x]{inputenc}
\usepackage[T1]{fontenc}
\usepackage[english]{babel}
\usepackage{amsmath, amssymb, amstext}
\usepackage{wrapfig}
\usepackage{float}
\usepackage{graphicx}
\usepackage{color}

\usepackage{mathtools}
\usepackage{units}

\usepackage{caption}
\usepackage{subcaption}

\bibliographystyle{unsrt}

\setlength{\parindent}{0pt}

\newcommand{\todo}[1]{\textcolor{red}{TODO: #1}}


\title{Master Thesis Notizen}
\author{Jan Lammel}
\date{\today{}, Heidelberg}



\begin{document}

\maketitle \ \\ 
\newpage

\section{Derivations}
\subsection{Heat equation with temperature dependency of $c_p$, $\rho$ and $\lambda$}


$T = T(\vec{x},t)$ but skipped in the following derivation for clarity.

\begin{align*}
	\frac{\partial T}{\partial t} = & \nabla \left[ a(\vec{x},T) \cdot \nabla T(\vec{x},T) \right] \\
	= & \nabla \left[ \frac{\lambda(T)}{\rho(T) \cdot c_p(T)} \nabla T \right] \\
	= & \nabla \left( \frac{\lambda(T)}{\rho(T) \cdot c_p(T)} \right) \nabla T + \frac{\lambda(T)}{\rho(T) \cdot c_p(T)} \Delta T \\
	= & \frac{1}{\rho (T) \cdot c_p (T)} \nabla \lambda (T) \nabla T + \lambda (T) \nabla \cdot \left[ \rho (T) \cdot c_p (T) \right]^{-1} \nabla T + \frac{\lambda (T)}{\rho (T) \cdot c_p (T)} \Delta T \\
	= & \frac{1}{\rho (T) \cdot c_p (T)} \nabla \lambda (T) \nabla T - \left[ \rho (T) \cdot c_p (T) \right]^{-2} \nabla \left[ \rho (T) \cdot c_p (T) \right] \lambda(T) \nabla T + \frac{\lambda (T)}{\rho (T) \cdot c_p (T)} \Delta T \\
	= & \frac{1}{\rho (T) \cdot c_p (T)} \nabla \lambda (T) \nabla T - \frac{\lambda (T)}{(\rho(T) \cdot c_p(T))^2} (\nabla T)^T \left[ \nabla \rho(T) c_p(T) + \rho(T) \nabla c_p(T) \right] \\
	& + \frac{\lambda (T)}{\rho (T) \cdot c_p (T)} \Delta T \\
	= & \frac{1}{\rho (T) \cdot c_p (T)} \frac{\partial \lambda}{\partial T} (\nabla T)^T \nabla T - \frac{\lambda (T)}{(\rho(T) \cdot c_p(T))^2} (\nabla T)^T \left[ \frac{\partial \rho}{\partial T} \nabla T \cdot c_p(T) + \rho(T) \frac{ \partial c_p}{\partial T} \nabla T \right] \\
	& + \frac{\lambda (T)}{\rho (T) \cdot c_p (T)} \Delta T \\
	= & \frac{1}{\rho (T) \cdot c_p (T)} \frac{\partial \lambda}{\partial T} (\nabla T)^T \nabla T - \frac{\lambda (T)}{(\rho(T))^2 c_p(T)} \frac{\partial \rho}{\partial T} (\nabla T)^T \nabla T - \frac{\lambda (T)}{\rho(T) (c_p(T))^2} \frac{\partial c_p}{\partial T} (\nabla T)^T \nabla T \\
	& + \frac{\lambda (T)}{\rho (T) \cdot c_p (T)} \Delta T \\
	= & \left[ \frac{1}{\rho (T) \cdot c_p (T)} \frac{\partial \lambda}{\partial T} - \frac{\lambda (T)}{(\rho(T))^2 c_p(T)} \frac{\partial \rho}{\partial T} - \frac{\lambda (T)}{\rho(T) (c_p(T))^2} \frac{\partial c_p}{\partial T} \right] \cdot (\nabla T)^T \nabla T + \frac{\lambda (T)}{\rho (T) \cdot c_p (T)} \Delta T \\
\end{align*}

\subsection{Discretization 1D, $\lambda$ const. $\forall T$}

\begin{equation*}
	(\nabla T)^k = \frac{T^{k+1} - T^k}{x^{x+1} - x^k} \underbrace{=}_{\substack{\text{equidistant} \\ \text{grid}}} =  \frac{T^{k+1} - T^k}{\Delta x}
\end{equation*}

\begin{equation*}
	(\Delta T)^k = \frac{T^{k+1} - 2 T^k + T^{k-1}}{(\Delta x)^2}
\end{equation*}

Boundary conditions:

\begin{itemize}
	\item $T^0(t) = T_{\text{oven}}(t)$ \quad with \quad $\dot{ T}_{\text{oven}}= \beta = const.$
	\item $(\nabla T)^{N-1} = \frac{T^{N} - T^{N-1}}{\Delta x} \stackrel{!}{=} 0 \quad \Leftrightarrow \quad T^{N-1} = T^N$ 
\end{itemize}

ODE-system with line-method:

\begin{align*}
	\dot{T}^k = \lambda & \left[ - \frac{1}{\rho(T^k) (c_p(T^k))^2} \cdot \frac{\partial c_p(T^k)}{\partial T^k} \cdot \left( \frac{T^{k+1} - T^k}{\Delta x} \right)^2 \right. \\
	& \left. - \frac{1}{(\rho(T^k))^2 c_p(T^k)} \cdot \frac{\partial \rho(T^k)}{\partial T^k} \cdot \left( \frac{T^{k+1} - T^k}{\Delta x} \right)^2 \right. \\
	& \left. + \frac{1}{c_p(T^k)} \cdot \frac{T^{k+1} - 2 T^k + T^{k-1}}{(\Delta x)^2} \right] \\
	& & &\text{for } k = 1, ..., N-1 \\
	T^0 & = T_{\text{oven}} \\
	T^N & = T^{N-1}
\end{align*}

which is equivalent to

\begin{align*}
	\hspace{-6cm}
	\frac{d}{dt} \begin{bmatrix*}
	T^0 \\
	T^1 \\
	T^2 \\
	\vdots \\
	T^{N_1} \\
	T^{N_1+1} \\
	\vdots \\ 
	T^{N_1+N_3-1}
	\end{bmatrix*} =
	\begin{bmatrix}
		\beta \\
		0 \\
		0 \\
		0 \\
		\vdots \\
		0 \\
		0 \\
		0
	\end{bmatrix}
	& -
	\begin{bmatrix}
	0 \\
	\frac{\lambda}{(\rho(T^1))^2 c_p(T^1)} \frac{\partial \rho (T^1)}{\partial T^1} \frac{(T^2 - T^1)^2}{(\Delta x)^2} \\
	\vdots \\
	\frac{\lambda}{(\rho(T^{N_1-1}))^2 c_p(T^{N_1-1})} \frac{\partial \rho (T^{N_1-1})}{\partial T^{N_1-1}} \frac{(T^{N_1} - T^{N_1-1})^2}{(\Delta x)^2} \\
	\frac{\lambda}{(\rho(T^{N_1}))^2 c_p(T^{N_1})} \frac{\partial \rho (T^{N_1})}{\partial T^{N_1}} \frac{(T^{N_1+1} - T^{N_1})^2}{(\Delta x)^2} \\
	\vdots \\
	\frac{\lambda}{(\rho(T^{N_1+N_3-2}))^2 c_p(T^{N_1+N_3-2})} \frac{\partial \rho (T^{N_1+N_3-2})}{\partial T^{N_1+N_3-2}} \frac{(T^{N_1+N_3-1} - T^{N_1+N_3-2})^2}{(\Delta x)^2} \\	
	0 \\			
	\end{bmatrix} \\
	& -
	\begin{bmatrix}
		0 \\
		\frac{\lambda}{\rho(T^1) (c_p(T^1))^2} \frac{\partial c_p (T^1)}{\partial T^1} \frac{(T^2 - T^1)^2}{(\Delta x)^2} \\
		\vdots \\
		\frac{\lambda}{\rho(T^{N_1-1}) (c_p(T^{N_1-1}))^2} \frac{\partial c_p (T^{N_1-1})}{\partial T^{N_1-1}} \frac{(T^{N_1} - T^{N_1-1})^2}{(\Delta x)^2} \\
		\frac{\lambda}{\rho(T^{N_1}) (c_p(T^{N_1}))^2} \frac{\partial c_p (T^{N_1})}{\partial T^{N_1}} \frac{(T^{N_1+1} - T^{N_1})^2}{(\Delta x)^2} \\
		\vdots \\
		\frac{\lambda}{\rho(T^{N_1+N_3-2}) (c_p(T^{N_1+N_3-2}))^2} \frac{\partial c_p (T^{N_1+N_3-2})}{\partial T^{N_1+N_3-2}} \frac{(T^{N_1+N_3-1} - T^{N_1+N_3-2})^2}{(\Delta x)^2} \\	
		0 \\			
	\end{bmatrix} \\
	& + 
	\begin{bmatrix}
	0 \\
	\frac{\lambda^{Const}}{c_p^{Const}(T^{1}) \cdot \rho^{Const}(T^{1})} \\
	\vdots \\
	\frac{\lambda^{Const}}{c_p^{Const}(T^{N_1-2}) \cdot \rho^{Const}(T^{N_1-2})} \\
	\frac{\lambda^{Const}}{c_p^{Const}(T^{N_1-1}) \cdot \rho^{Const}(T^{N_1-1})} \\ 
	\frac{\lambda^{PCM}}{c_p^{PCM}(T^{N_1}) \cdot \rho^{PCM}(T^{N_1})} \\
	\vdots \\
	\frac{\lambda^{PCM}}{c_p^{PCM}(T^{N_1+N_3-2}) \cdot \rho^{PCM}(T^{N_1+N_3-2})} \\
	\frac{\lambda^{PCM}}{c_p^{PCM}(T^{N_1+N_3-1}) \cdot \rho^{PCM}(T^{N_1+N_3-1})}
	\end{bmatrix}^T
	\cdot
	\begin{bmatrix}
	0 \\
	 \frac{T^{0} - 2 T^{1} + T{2}}{\Delta x_{\text{Const}}^2} \\
	\vdots \\
	\frac{T^{N_1-3} - 2 T^{N_1-2} + T{N_1-1}}{\Delta x_{\text{Const}}^2} \\
	\frac{\frac{2}{1+\alpha} T^{N_1-2} - \frac{2}{\alpha} T^{N_1-1} + \frac{2}{\alpha (\alpha + 1)} T{N_1}}{\Delta x_{\text{Const}}^2} \\
	\frac{T^{N_1-1} - 2 T^{N_1} + T{N_1+1}}{\Delta x_{\text{PCM}}^2} \\
	\vdots \\
	\frac{T^{N_1+N_3-3} - 2 T^{N_1+N_3-2} + T{N_1+N_3-1}}{\Delta x_{\text{PCM}}^2} \\
	\frac{T^{N_1+N_3-2} - T^{N_1+N_3-1}}{\Delta x_{\text{PCM}}^2}
	\end{bmatrix}
\end{align*}


\section{Parameter Estimation}

Measurements: $U_{dsc}(T_{ref};\beta)$ \\
In the experiment we have a temperature difference $\Delta T(T_{ref})$ between sample reference. 
Due to the Seebeck-effect we can measure this temperature difference as an electric potential $U_{dsc}(T_{ref})$. \\


Simulation: $\Delta T = T_{ref} - T_{PCM}$ \\
In order to get a temperature difference we perform the integration twice. 
In both cases there is the Constantan disc to transport heat to the position where the crucible in the experiment is. 
Now in one integration at the end of the Constantan there is the PCM. 
In the reference integration there is no PCM, just the boundary conditions of no heat-flux at the end of the Constantan.
The temperature difference is taken in both cases at the very end of the Constantan disc. \\

There is a mapping $f$ from temperature difference to electric potential: $\Delta U = f(\Delta T)$.
For simplicity we assume a linear relationship $\Delta U = k \cdot \Delta T(T_{ref})$ with parameter $k$ to optimize because we do not know this constant. \\

\subsection{Optimization Problem}
\subsubsection{DSC voltage fit}

In order to obtain the specific heat capacity $c_p$ (and $k$) we solve the least square problem where we minimize the residuum of the measured and simulated voltages.

\begin{align*}
	& \min_{p_{c_p}, k} ||U_{dsc} - \Delta U ||_2^2 \\
	\Leftrightarrow & \min_{p_{c_p}, k} ||U_{dsc} - k (T_{ref} - T_{pcm}) ||_2^2 \\
\end{align*}


\subsubsection{DSC heat flux fit}
So as to avoid the unknown proportional factor $k$ we now minimize the residuum of the measured and simulated heat flux into the PCM. Therefore we are using Fourier's law

\begin{equation}
	q = - \lambda \cdot \nabla T
\end{equation}
\label{eq:def_fouriers_law}

where $q$ is the heat flux density, $\lambda$ the heat conductivity. So the total heat flux through an area element $A$ is

\begin{equation*}
	\Phi_q = q \cdot A =  - \lambda \cdot \nabla T \cdot A
\end{equation*}

Furthermore we use the definition of heat capacity

\begin{equation}
	C = \frac{dQ}{dT} \simeq \frac{\Delta Q}{\Delta T}
\end{equation}
\label{eq:def_heat_capacity}

i.e. transfered heat $Q$ over a infinitesimal temperature change $T$. \\

Assuming constant pressure we can write for the heat change of a volume element $i$

\begin{equation*}
	\dot{Q}_i \stackrel{\cdot}{=} \frac{\Delta Q_i}{\Delta t} = \frac{c_p \cdot m_i \cdot \Delta T}{\Delta t}
\end{equation*}

Using the continuity equation we get the relation

\begin{equation*}
	\dot{Q}_i = \Phi_i^{\text{in}} - \Phi_i^{\text{out}}
\end{equation*}

with which we get the wanted heat flux into the first PCM segment:

\begin{align*}
	\Phi_{\text{PCM,1}}^{\text{in}} = & \dot{Q}_{\text{PCM,1}} + \Phi_{\text{PCM,1}}^{\text{out}} \\
	= & \frac{c_p \cdot m_{\text{PCM,}1} \cdot \Delta T}{\Delta t} - \lambda \cdot \nabla T \cdot A \\
	= & c_p \cdot m_{\text{PCM,}1} \frac{T_{\text{PCM,}1}^{k+1} - T_{\text{PCM,}1}^k}{\Delta t} - \frac{\lambda \ m_i}{\rho \ \Delta x_i} \cdot \frac{T_{\text{PCM,}2}^{k} - T_{\text{PCM,}1}^k}{\Delta x_i}
\end{align*}

using $m_i = \rho \ \Delta x_i \ A \Leftrightarrow A = \frac{m_i}{\rho \ \Delta x_i}$

\todo{Mechanische Arbeit hier bisher nicht beruecksichtigt weil sich $\rho$ ja schon betraechtlich aendert...?}

An alternative way of computing $\Phi_{\text{PCM,}1}^{\text{in}}$ is to sum up all heat changes within the PCM:

\begin{align*}
	\Phi_{\text{PCM,1}}^{\text{in}} = \sum\limits_{i=\text{PCM,1}}^{\text{PCM,}N_3} \frac{c_p(T_i) \ m_i \ \Delta T_i}{\Delta t}
\end{align*}


\section{Explicit functions for $c_p$ and $\rho$-fit}
\subsection{Own $\arctan$ function}

\begin{equation*}
	c_p(T) = \left[ \arctan (-p_3 (T-p_0)) + \frac{\pi}{2} \right] \cdot \left[ p_1 \exp(-p_2(T-p_0)^2)] \right] + p_4 \cdot T + p_5 
\end{equation*}


\begin{equation*}
	c_p(T) = \left[ \frac{1}{\exp(p_3(T-p_0)) + 1} \right] \cdot \left[ p_1 \exp(-p_2(T-p_0)^2)] \right] + p_4 \cdot T + p_5 
\end{equation*}

\subsection{Fraser-Suzuki Peak}

\begin{align}
	f(x) = \left\{ \begin{array}{lr}
	 h \cdot \exp\left[ -\frac{\ln(r)}{\ln(sr)^2} \left( \ln(1+(x-z) \cdot \frac{sr^2 - 1}{wr \cdot sr}) \right)^2 \right] & \text{for } x < z - \frac{wr \cdot sr}{sr^2 - 1} \\
	 0 & \text{else}
	\end{array} \right.
\end{align}



\section{Related work}
\subsection{Applications}
\begin{itemize}
	\item PCM added to photovoltaic cells to smooth temperature conditions -> higher efficiency \cite{pv-pcm}
	\item Energy conversation in buildings -> Passivhaeuser 
\end{itemize}

\subsection{PCM improvements}
\begin{itemize}
	\item added CNT pellets to increase thermal conductivity \cite{cnt_pellets}
\end{itemize}



\newpage
\begin{thebibliography}{9}

  
\bibitem{pv-pcm}
  Yearly energy performance of a photovoltaic-phase change material
  (PV-PCM) system in hot climate, 
  \textit{A. Hasan, J. Sarwar, H. Alnoman, S. Abdelbaqi},
  2017,
  Solar Energy, Volume 146, April 2017, Pages 417-429

 \bibitem{cnt_pellets}
   Heat transfer performance enhancement of paraffin/expanded
   perlite phase change composites with graphene nano-platelets,
   \textit{Sayanthan Ramakrishnan, Xiaoming Wang, Jay Sanjayan, John Wilson},
   2017
  
  
\end{thebibliography}

\end{document}